
\section{Exercises}

Use an index card or a piece of paper folded lengthwise, and cover up the
right-hand column of the exercises below. Read each exercise in the
left-hand column, answer it in your mind, then slide the index card down to
reveal the answer and see if you're right! For every exercise you missed,
figure out why you missed it before moving on.

\begin{small}
\begin{enumerate}
\newcolumntype{Q}{>{\arraybackslash}m{.4\textwidth}}
\newcolumntype{A}{>{\arraybackslash}m{.5\textwidth}}
%\begin{longtable}{m{0.3\textwidth} || m{0.6\textwidth}}
\begin{longtable}{Q || A}
\hline
\item Is the set \{~Will, Smith~\} the same as the set \{~Smith, Will~\}?
&
Yes indeed.
\\
\hline

\item Is the ordered pair (Will, Smith) the same as (Smith, Will)?
&
No. Order matters with ordered pairs (hence the name), and with any size 
tuple for that matter.
\\
\hline

\item Is the set \{~\{~Luke, Leia~\}, Han~\} the same as the set \{~Luke,
\{~Leia, Han~\}~\}?
&
No. For instance, the first set has Han as a member but the second set does
not. (Instead, it has another set as a member, and that inner set happens
to include Han.)
\\
\hline

\item What's the first element of the set \{~Cowboys, Redskins, Steelers~\}?
&
The question doesn't make sense. There is no ``first element" of a set. All
three teams are equally members of the set, and could be listed in any
order.
\\
\hline

\item \label{definesets1} Let $G$ be \{~Matthew, Mark, Luke, John~\}, $J$
be \{~Luke, Obi-wan, Yoda~\}, $S$ be the set of all Star Wars characters, and
$F$ be the four gospels from the New Testament.

Now then. Is $J \subseteq G$?
&
No.
\\
\hline

\item Is $J \subseteq S$?
&
Yes.
\\
\hline

\item Is Yoda $\in J$?
&
Yes.
\\
\hline

\item Is Yoda $\subseteq J$?
&
No. Yoda isn't even a set, so it can't be a subset of anything.
\\
\hline

\item Is \{~Yoda~\} $\subseteq J$?
&
Yes. The (unnamed) set that contains only Yoda is in fact a subset of $J$.
\\
\hline

\item Is \{~Yoda~\} $\in J$?
&
No. Yoda is one of the elements of $J$, but \{~Yoda~\} is not. In other
words, $J$ contains Yoda, but $J$ does not contain a set
which contains Yoda (nor does it contain any sets at all, in fact).
\\
\hline

\item Is $S \subseteq J$?
&
No.
\\
\hline

\item Is $G \subseteq F$?
&
Yes, since the two sets are equal.
\\
\hline

\item Is $G \subset F$?
&
No, since the two sets are equal, so neither is a \textit{proper} subset of
the other.
\\
\hline

\item Is $\varnothing \subseteq S$?
&
Yes, since the empty set is a subset of \textit{every} set.
\\
\hline

\item Is $\varnothing \subseteq \varnothing$?
&
Yes, since the empty set is a subset of \textit{every} set.
\\
\hline

\item Is $F \subseteq \Omega$?
&
Yes, since every set is a subset of $\Omega$.
\\
\hline

\item Is $F \subset \Omega$?
&
Yes, since every set is a subset of $\Omega$, and $F$ is certainly not
equal to $\Omega$.
\\
\hline

\item Suppose $X$ = \{ Q, $\varnothing$, \{ Z \} \}. Is $\varnothing \in X$?
Is $\varnothing \subseteq X$?
&
Yes and yes. The empty set is an element of $X$ because it's one of the
elements, and it's also a subset of $X$ because it's a subset of every set.
Hmmm.
\\
\hline

\item \label{definesets2} Let $A$ be \{~Macbeth, Hamlet, Othello~\}, $B$ be
\{~Scrabble, Monopoly, Othello~\}, and $T$ be \{~Hamlet, Village, Town~\}.

What's $A \cup B$?
&
\{~Macbeth, Hamlet, Othello, Scrabble, Monopoly~\}. (The elements can be
listed in any order.)
\\
\hline

\item What's $A \cap B$?
&
\{~Othello~\}.
\\
\hline

\item What's $A \cap \overline{B}$?
&
\{~Macbeth, Hamlet~\}.
\\
\hline

\item What's $B \cap T$?
&
$\varnothing$.
\\
\hline

\item What's $B \cap \overline{T}$?
&
$B$. (which is \{~Scrabble, Monopoly, Othello~\}.)
\\
\hline

\item What's $A \cup (B \cap T)$? \label{bintersecttinparens}
&
\{~Hamlet, Othello, Macbeth~\}.
\\
\hline

\item What's $(A \cup B) \cap T$?
&
\{~Hamlet~\}. (Note: not the same answer as in item \ref{bintersecttinparens} now that the parens are placed differently.)
\\
\hline

\item What's $A - B$?
&
\{~Macbeth, Hamlet~\}.
\\
\hline

\item What's $T - B$?
&
Simply $T$, since the two sets have nothing in common.
\\
\hline

\item What's $T \times A$?
&
\{~(Hamlet, Macbeth), (Hamlet, Hamlet), (Hamlet, Othello), 
   (Village, Macbeth), (Village, Hamlet), (Village, Othello), 
   (Town, Macbeth), (Town, Hamlet), (Town, Othello)~\}. The order of the
ordered pairs within the set is not important; the order of the elements within
each ordered pair \textit{is} important.
\\
\hline

\item What's $(B \cap B) \times (A \cap T)$?
&
\{~(Scrabble, Hamlet), (Monopoly, Hamlet), (Othello, Hamlet)~\}.
\\
\hline

\item What's $|A \cup B \cup T|$?
&
7.
\\
\hline

\item What's $|A \cap B \cap T|$?
&
0.
\\
\hline

\item What's $|(A \cup B \cup T) \times (B \cup B \cup B)|$?
&
21. (The first parenthesized expression gives rise to a set with 7
elements, and the second to a set with three elements ($B$ itself). Each
element from the first set gets paired with an element from the second, so
there are 21 such pairings.)
\\
\hline

\item Is $A$ an extensional set, or an intensional set?
&
The question doesn't make sense. Sets aren't ``extensional" or
``intensional"; rather, a given set can be \textit{described} extensionally
or intensionally. The description given in item~\ref{definesets2} is an
extensional one; an intensional description of the same set would be ``The
Shakespeare tragedies Stephen studied in high school."
\\
\hline

\item Recall that $G$ was defined as \{~Matthew, Mark, Luke, John~\}. Is
this a partition of $G$?
\begin{itemize}
\item \{~Luke, Matthew~\}
\item \{~John~\}
\end{itemize}
&
No, because the sets are not collectively exhaustive (Mark is missing).
\\
\hline

\item Is this a partition of $G$?
\begin{itemize}
\item \{~Mark, Luke~\}
\item \{~Matthew, Luke~\}
\end{itemize}
&
No, because the sets are neither collectively exhaustive (John is missing)
nor mutually exclusive (Luke appears in two of them).
\\
\hline

\item Is this a partition of $G$?
\begin{itemize}
\item \{~Matthew, Mark, Luke~\}
\item \{~John~\}
\end{itemize}
&
Yes. (Trivia: this partitions the elements into the synoptic gospels and
the non-synoptic gospels).
\\
\hline

\item Is this a partition of $G$?
\begin{itemize}
\item \{~Matthew, Luke~\}
\item \{~John, Mark~\}
\end{itemize}
&
Yes. (This partitions the elements into the gospels which feature a
Christmas story and those that don't).
\\
\hline

\item Is this a partition of $G$?
\begin{itemize}
\item \{~Matthew, John~\}
\item \{~Luke~\}
\item \{~Mark~\}
\item $\varnothing$
\end{itemize}
&
Yes. (This partitions the elements into the gospels that were written by
Jews, those that were written by Greeks, those that were written by
Romans, and those that were written by Americans).
\\
\hline

\item What's the power set of \{~Rihanna~\}?
&
\{ \{ Rihanna \}, $\varnothing$ \}.
\\
\hline

\item Is \{~peanut, jelly~\} $\in \mathbb{P}$(\{~peanut, butter, jelly~\}?
&
Yes, since \{~peanut, jelly~\} is one of the eight subsets of \{~peanut,
butter, jelly~\}. (Can you name the other seven?)
\\
\hline

\item Is it true for \textit{every} set $S$ that $S \in \mathbb{P}(S)$?
&
Yep.
\\
\hline

\end{longtable}
\end{enumerate}
\end{small}
