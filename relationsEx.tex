
\section{Exercises}

\begin{small}
\begin{enumerate}
\newcolumntype{Q}{>{\arraybackslash}m{.4\textwidth}}
\newcolumntype{A}{>{\arraybackslash}m{.5\textwidth}}
%\begin{longtable}{m{0.3\textwidth} || m{0.6\textwidth}}
\begin{longtable}{Q || A}
\hline
\item Let $A$ be the set \{~Chuck, Julie, Sam~\} and $S$ be the set
\{~basketball, volleyball~\}. \\
Is \{~(Julie, basketball), (Sam, basketball), (Julie, volleyball)~\} a
relation between $A$ and $S$?
&
Yes it is, since it is a subset of $A \times S$.
\\
\hline
\item Is the above relation an endorelation?
&
No, because an endorelation involves one set with itself, not two different
sets (like $A$ and $S$ are.)
\\
\hline
\item Is \{~(Chuck, basketball), (basketball, volleyball)~\} a
relation between $A$ and $S$?
&
No, since the first element of one of the ordered pairs is not from the set
$A$.
\\
\hline
\item Is $\varnothing$ a relation between $A$ and $S$?
&
Yes it is, since it is a subset of $A \times S$.
\\
\hline
\item How large could a relation between $A$ and $S$ be?
&
The maximum cardinality is 6, if all three athletes played all three
sports. (I'm assuming that the meaning of the relation is ``\textsl{plays}"
instead of ``\textsl{isAFanOf}" or ``\textsl{knowsTheRulesFor}" or something
else. In any case, the maximum cardinality is 6.)
\\
\hline
\item Let $T$ be the set \{~Spock, Kirk, McCoy, Scotty, Uhura~\}. Let
$O$ be an endorelation on $T$, defined as follows: \{~(Kirk, Scotty),
(Spock, Scotty), (Kirk, Spock), (Scotty, Spock)~\}. \\
Is $T$ reflexive?
&
No, since it doesn't have any of the elements of $T$ appearing with
themselves.
\\
\hline
\item Is $T$ symmetric?
&
No, since it contains (Kirk, Scotty) but not (Scotty, Kirk).
\\
\hline
\item Is $T$ antisymmetric?
&
No, since it contains (Spock, Scotty) and also (Scotty, Spock).
\\
\hline
\item Is $T$ transitive?
&
Yes, since for every $(x,y)$ and $(y,z)$ present, the corresponding $(x,z)$
is also present. (The only example that fits this is $x$=Kirk, $y$=Spock,
$z$=Scotty, and the required ordered pair is indeed present.)
\\
\hline
\item Let $H$ be an endorelation on $T$, defined as follows: \{~(Kirk, Kirk),
(Spock, Spock), (Uhura, Scotty), (Scotty, Uhura), (Spock, McCoy), (McCoy,
Spock), (Scotty, Scotty), (Uhura, Uhura)~\}. \\
Is $H$ reflexive?
&
No, since it's missing (McCoy, McCoy).
\\
\hline
\item Is $H$ symmetric?
&
Yes, since for every $(x,y)$ it contains, the corresponding $(y,x)$ is also
present.
\\
\hline
\item Is $H$ antisymmetric?
&
No, since it contains (Uhura, Scotty) and also (Scotty, Uhura).
\\
\hline
\item Is $H$ transitive?
&
Yes, since there aren't any examples of $(x,y)$ and $(y,z)$ pairs both
being present.
\\
\hline
\item Let \textsl{outranks} be an endorelation on the set of all crew members of
the Enterprise, where $(x,y) \in$ \textsl{outranks} if character $x$ has a
higher Star Fleet rank than $y$. \\
Is \textsl{outranks} reflexive?
&
No, since no officer outranks him/herself.
\\
\hline
\item Is \textsl{outranks} symmetric?
&
No, since an officer cannot outrank an officer who in turn outranks
him/her.
\\
\hline
\item Is \textsl{outranks} antisymmetric?
&
Yes, since if one officer outranks a second, the second one \textit{cannot}
also outrank the first.
\\
\hline
\item Is \textsl{outranks} transitive?
&
Yes, since if one officer outranks a second, and that officer outranks a
third, the first obviously also outranks the third.
\\
\hline
\item Is \textsl{outranks} a partial order?
&
No, but close. It satisfies antisymmetry and transitivity, which are
crucial. The only thing it doesn't satisfy is reflexivity, since none of
the members appear with themselves. If we changed this relation to
\textsl{ranksAtLeastAsHighAs}, then we could include these ``double" pairs
and have ourselves a partial order.
\\
\hline
\item Let \textsl{sameShirtColor} be an endorelation on the set of all crew
members of the Enterprise, where $(x,y) \in$ \textsl{sameShirtColor} if
character $x$ ordinarily wears the same shirt color as character $y$. \\ Is
\textsl{sameShirtColor} reflexive?
&
Yes, since you can't but help wear the same shirt color as you're wearing.
\\
\hline
\item Is \textsl{sameShirtColor} symmetric?
&
Yes, since if a crew member wears the same shirt color as another, then that
second crew member also wears the same shirt color as the first. If Scotty
and Uhura both wear red, then Uhura and Scotty both wear red, duh.
\\
\hline
\item Is \textsl{sameShirtColor} antisymmetric?
&
No, for probably obvious reasons.
\\
\hline
\item Is \textsl{sameShirtColor} transitive?
&
Yes. If Kirk and Sulu wear the same color (yellow), and Sulu and Chekov
wear the same color (yellow), then Kirk and Chekov most certainly will
wear the same color (yellow).
\\
\hline
\item Above, we defined $A$ as the set \{~Chuck, Julie, Sam~\} and $S$ as the set
\{~basketball, volleyball~\}. Then we defined the relation \{~(Julie,
basketball), (Sam, basketball), (Julie, volleyball)~\}. \\
Is this relation a function?
&
No, because it's missing Chuck entirely.
\\
\hline
\item Suppose we added the ordered pair (Chuck, basketball) to it. Now is it a
function?
&
No, because Julie appears twice, mapping to two different values.
\\
\hline
\item Okay. Suppose we then remove (Julie, volleyball). We now have \{~(Julie,
basketball), (Sam, basketball), (Chuck, basketball)~\}. Is \textit{this} a
function?
&
Yes. Congratulations.
\\
\hline
\item Let's call this function ``\textsl{faveSport}," which suggests that its
meaning is to indicate which sport is each athlete's favorite. 
What's the domain of faveSport?
&
\{~Julie, Chuck, Sam~\}.
\\
\hline
\item What's the codomain of faveSport?
&
\{~basketball, volleyball~\}.
\\
\hline
\item What's the range of faveSport?
&
\{~basketball~\}.
\\
\hline
\item Is faveSport injective?
&
No, because Julie and Sam (and Chuck) all map to the same value
(basketball). For a function to be injective, there must be no two domain
elements that map to the same codomain element.
\\
\hline
\item Is there any way to make it injective?
&
Not without altering the underlying sets. There are three athletes and two
sports, so we can't help but map multiple athletes to the same sport.
\\
\hline
\item Fine. Is faveSport surjective?
&
No, because no one maps to volleyball.
\\
\hline
\item Is there any way to make it surjective?
&
Sure, for instance change Sam from basketball to volleyball. Now both of
the codomain elements are ``reachable" by some domain element, so it's
surjective.
\\
\hline
\item Is faveSport now also bijective?
&
No, because it's still not injective.
\\
\hline
\item How can we alter things so that it's bijective?
&
One way is to add a third sport --- say, kickboxing --- and move either
Julie or Chuck over to kickboxing. If we have Julie map to kickboxing, Sam
map to volleyball, and Chuck map to basketball, we have a bijection.
\\
\hline
\item How do we normally write the fact that ``Julie maps to kickboxing"?
&
faveSport(Julie) = kickboxing.
\\
\hline
\item What's another name for ``injective?"
&
one-to-one.
\\
\hline
\item What's another name for ``surjective?"
&
onto.
\\
\hline
\item What's another name for ``range?"
&
image.
\\
\hline


\end{longtable}
\end{enumerate}
\end{small}
