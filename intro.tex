
\chapter{Meetup at the trailhead}

Before we set out on our ``cool, brisk walk," let's get oriented. What
\textit{is} discrete mathematics, anyway? Why is it called that? What does
it encompass? And what is it good for?

Let's take the two words of the subject, in reverse order. First,
\textit{math}. When most people hear ``math," they think ``numbers." After
all, isn't math the study of quantity? And isn't that the class where we
first learned to count, add, and multiply?

Mathematics certainly has its root in the study of numbers ---
specifically, the ``natural numbers" (the integers from 1 on up) that
fascinated the ancient Greeks. Yet math is broader than this, almost to the
point where numbers can be considered a special case of something deeper.
In this book, when we talk about trees, sets, or formal logic, there might
not be a number in sight.

Math is about \textbf{abstract, conceptual objects that have
properties, and the implications of those properties.} An ``object" can be
any kind of ``thought material" that we can define and reason about
precisely. Much of math deals with questions like, ``suppose we defined a
certain kind of thing that had certain attributes.  What would be the
implications of this, if we reasoned it all the way out?" The ``thing" may
or may not be numerical, whatever it turns out to be. Like a number,
however, it will be crisply defined, have certain known aspects to it, and
be capable of combining with other things in some way.

Fundamental to math is that it deals with the \textit{abstract}. Abstract,
which is the opposite of concrete, essentially means something that can't
be perceived with the senses. A computer chip is concrete: you can touch
it, you can see it. A number is not; nor is a function, a binary tree, or a
logical implication. The only way to perceive these things is with the
power of the mind. We will write expressions and draw pictures of many of
our mathematical structures in order to help visualize them, and nearly
everything we study will have practical applications whereby the
abstractness gets grounded in concreteness for some useful purpose. But the
underlying mathematical entity remains abstract and ethereal --- only
accessible to the mind's eye. We may use a pencil to form the figure ``5"
on a piece of paper, but that is only a concrete manifestation of the
underlying concept of ``five-ness." Don't mistake the picture or the symbol
for the thing itself, which always transcends any mere physical
representation.

The other word in the name of our subject is ``discrete" (not to be
confused with ``discreet," which means something else entirely). The best
way to appreciate what discrete means is to contrast it with its opposite,
continuous. Consider the following list:

\begin{center}
\begin{tabular}{c c}
Discrete & Continuous \\
\hline
whole numbers ($\mathbb{Z}$) & real numbers ($\mathbb{R}$) \\
\texttt{int} & \texttt{double} \\
digital & analog \\
quantum & continuum \\
counting & measuring \\
number theory & analysis \\
$\Sigma$ & $\int$ \\
-- & $\frac{d}{dx}$ \\
\end{tabular}
\end{center}

What do the left-hand entries have in common? They describe things that are
measured in crisp, distinct intervals, rather than varying smoothly over a
range. Discrete things jump suddenly from position to position, with rigid
precision.  If you're 5 feet tall, you might some day grow to 5.3 feet; but
though there might be 5 people in your family, there will never be 5.3
(although there could be 6 someday).

The last couple of entries on this list are worth a brief comment. They are
math symbols, some of which you may be familiar with. On the right side ---
in the continuous realm --- are $\int$ and $\frac{d}{dx}$, which you'll
remember if you've taken calculus. They stand for the two fundamental
operations of integration and differentiation. Integration, which can be
thought of as finding ``the area under a curve," is basically a way of
adding up a whole infinite bunch of numbers over some range. When you
``integrate the function $x^2$ from 3 to 5," you're really adding up all
the tiny, tiny little vertical slivers that comprise the area from $x=3$ on
the left to $x=5$ on the right. Its corresponding entry in the left-column
of the table is $\Sigma$, which is just a short-hand for ``sum up a bunch
of things." Integration and summation are equivalent operations, it's just
that when you integrate, you're adding up all the (infinitely many) slivers
across the real-line continuum. When you sum, you're adding up a fixed
sequence of entries, one at a time, like in a loop. $\Sigma$ is just the
discrete ``version" of $\int$. 

The same sort of relationship holds between ordinary subtraction (``--")
and differentiation ($\frac{d}{dx}$). If you've plotted a bunch of discrete
points on $x$-$y$ axes, and you want to find the slope between two of them,
you just subtract their $y$ values and divide by the ($x$) distance between
them. If you have a smooth continuous function, on the other hand, you use
differentiation to find the slope at a point: this is essentially
subtracting the tiny tiny difference between two supremely close points and
then dividing by the distance between them. Thus subtraction is just the
discrete ``version" of $\frac{d}{dx}$.

Don't worry, you don't need to have fully understood any of the integration
or differentiation stuff I just talked about, or even to have taken
calculus yet. I'm just trying to give you some feel for what ``discrete"
means, and how the dichotomy between discrete and continuous really runs
through all of math and computer science. In this book, we will mostly be
focusing on discrete values and structures, which turn out to be of more
use in computer science.  That's partially because as you probably know,
computers themselves are discrete, and can only store and compute discrete
values. There can be many of them --- megabytes, gigabytes, terabytes ---
but each value stored is fundamentally comprised of bits, each of which has
a value of either 0 or 1. This is unlike the human brain, by the way, whose
neuronal synapses communicate based on the \textit{continuous} quantities
of chemicals present in their axons. So I guess ``computer" and ``brain"
are another pair of entries we could add to our discrete vs. continuous
list.

There's another reason, though, why discrete math is of more use to
computer scientists than continuous math is, beyond just the bits-and-bytes
thing. Simply put, computers operate algorithmically. They carry out
programs in step-by-step, iterative fashion. First do this, then do that,
then move on to something else. This mechanical execution, like the ticking
of a clock, permeates everything the computer can do, and everything we can
tell it to do. At a given moment in time, the computer \textit{has}
completed step 7, but \textit{not} step 8; it has accumulated 38 values,
but not yet 39; its database has exactly 15 entries in it, no more and no
less; it knows that after accepting this friend request, there will be
exactly 553 people in your set of friends. The whole paradigm behind
reasoning about computers and their programs is discrete, and that's why we
computer scientists find different problems worth thinking about than most
of the world did a hundred years ago.

But it's still math. It's just \textit{discrete} math. There's a lot to
come, so limber up and let me know when you're ready to hit the road.

\section{Exercises}

Use an index card or a piece of paper folded lengthwise, and cover up the
right-hand column of the exercises below. Read each exercise in the
left-hand column, answer it in your mind, then slide the index card down to
reveal the answer and see if you're right! For every exercise you missed,
figure out why you missed it before moving on.

\begin{small}
\begin{enumerate}
\newcolumntype{Q}{>{\arraybackslash}m{.3\textwidth}}
\newcolumntype{A}{>{\arraybackslash}m{.6\textwidth}}
%\begin{longtable}{m{0.3\textwidth} || m{0.6\textwidth}}
\begin{longtable}{Q || A}
\hline
\item 
What's the opposite of concrete?
&
Abstract.
\\
\hline
\item 
What's the opposite of discrete?
& 
Continuous.
\\
\hline
\item 
Consider a quantity of water in a glass. Would you call it abstract, or concrete?
Discrete, or continuous?
& 
Concrete, since it's a real entity you can experience with the senses.
Continuous, since it could be any number of ounces (or liters, or
tablespoons, or whatever). The amount of water certainly doesn't have to be
an integer. (Food for thought: since all matter is ultimately comprised of
atoms, are even substances like water discrete?)
\\
\hline
\item 
Consider the number 27. Would you call it abstract, or concrete?
Discrete, or continuous?
& 
Abstract, since you can't see or touch or smell ``twenty-seven." Probably
discrete, since it's an integer, and when we think of whole numbers we
think ``discrete." (Food for thought: in real life, how would you know
whether I meant the integer ``27" or the decimal number ``27.0?" And does
it matter?)
\\
\hline
\item 
Consider a bit in a computer's memory. Would you call it abstract, or
concrete?  Discrete, or continuous?
& 
Clearly it's discrete. Abstract vs. concrete, though, is a little tricky.
If we're talking about the actual transistor and capacitor that's
physically present in the hardware, holding a tiny charge in some little
chip, then it's concrete. But if we're talking about the value ``1" that is
conceptually part of the computer's currently executing state, then it's
really abstract just like 27 was. In this book, we'll always be talking
about bits in this second, abstract sense.
\\
\hline
\item 
If math isn't just about numbers, what else is it about?
& 
Any kind of abstract object that has properties we can reason about.
\\
\hline

\end{longtable}
\end{enumerate}
\end{small}
